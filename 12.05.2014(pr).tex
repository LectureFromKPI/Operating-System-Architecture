\subsection{Функціонування}\marginpar{\framebox{12.05.2014}}
Що буде на контрольній роботі з MIPS:
\begin{itemize}
\item описание команды, обрамление, машинный код;
\item описание алгоритма, комментарии к коду,  (10 б);
\item описать функционирование.
\end{itemize}

MIPS-lite інструкції, які можуть трапитися:
\begin{itemize}
\item ADD або SUB
\item OR I
\item LOAD або  STORE Word
\item BRANCH
\end{itemize}
\begin{tsk}
\begin{equation}
 lw \$t1, offset(\$t2);
\end{equation}
Команда завантажує у регістр \$t1 дані з пам’яті за адресою \$t2 з сувом offset.\\
\begin{tabular}{|c|c|c|}
\hline
№ 	& Микрооперация 			& Управляющий сигнал\\
\hline
1 	& PCout 					& IorD=0; \\
2 	& Цикл пам’яті; 				& MemRead=1\\
 3	& $ALU_{a}$:=(PC)				& AluSrcA=0\\
 4	& $ALU_b$:=(4)				& AluSrcB=01\\
 5	& ALU :=$ALU_a+ALU_b$			& $ALU_{op}$=00 -> $ALU_{control}$ = 0010( = 2)\\
 6	& ALUout := (ALU)			& PCSource=00;\\
 7	& PC:=(ALU)					&  PCWrite=1;\\
 8	& IR:=(($PC_{old}$))			& $IR_{wr}$=1;\\
 9  & CU[5-0]:=IR[31-26]        &  \\
 10 & DC           				& \\ 
 11 & A:=IR[25-21]              &\\ 
    &  B:=IR[20-16]              &
    \begin{tabular}{c} 9,10,11 пункт\\ виконуються майже одночасно.\\ Цими мікроопераціями мікропроцесор\\ готує майбутню операцію R типу \end {tabular}\\
 12 & $ALU_a$:=($PC_{new}$)           & $ALU_{srcA}$ = 0\\
 13 & $ALU_b$:=(SE(IR[15:0])?<<2	& $ALU_{srcB}$ = 11\\
 14 & ALU:=$(ALU_a)+(ALU_B) $     & \\
 15 & $ALU_{out}$=(ALU)   			& $ALU_{op}$ = 00 -> $ALU_{control}$ = 0010\\
 16 & $ALU_a$:= A 				& $ALU_{srcA}$ = 0\\
 17	& $ALU_b$:=SE(IR[15:0])       & $ALU_{scrB}$ = 10\\
 18	& ALU:=$(ALU_a) + (ALU_b) $   & $ALU_{op}$ = 00\\
 19	& $ALU_{out}$ := (ALU)          & $ALU_{control}$ = 0010\\
 20 & $M_{adress}$ := $(ALU_{out})$     & IorD = 1\\
 21 & ЦП 						& MemRead = 1 \\
 22 & MDR := (($ALU_{out}$))        & MemToReg = 1 	\\
 23 & ((IR[20-16])):=(MDR)		& RegWrite = 1 \\
 \hline
\end{tabular}
\end{tsk}
